\chapter{Brightkite Top 36 Users}
\label{appendix:brightike-top-36-users}

\begin{table}[ht]
    \centering
    \small
    \caption{36 most challenging users in the Brightkite dataset.}
    \csvautotabular{data/brightkite_top_36_users.csv}
    \label{tab:brightkite-top-36-users}
\end{table}

\newpage

From each user we have:

\begin{itemize}
    \item \textbf{User ID}: The unique identifier of the user in the original BrightKite dataset.
    \item \textbf{Number of Accesses}: The number of check-ins made by the user, this is used as the 
    item or page accesses in the multi-tenant caching problem.
    \item \textbf{Unique Items (Locations)}: The number of unique locations visited by the user, this is
    the number of unique items or pages in the multi-tenant caching problem.
    \item \textbf{Extra Cache Faults by Opt. Alg}: The number of extra cache faults that the optimal
    (clairvoyant) cache algorithm has with a cache size of 50. By extra it means the number of cache
    faults that are not inevitable, that is, all cache faults except the ones that are caused by the 
    first access to an item.
    \item \textbf{Rank}: The rank of the user in the dataset, based on the number of extra cache faults
    by the optimal cache algorithm.
\end{itemize}