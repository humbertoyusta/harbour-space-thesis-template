\chapter{Algorithm Overview}

\section{Base Algorithm}

Let's describe the base algorithm of this thesis for multi-tenant caching, its detailed 
pseudcode is shown in Figure \ref{fig:base_algorithm}.

For each page access, the algorithm first checks if the page is in the cache,
in case it is, it returns the buffer location and the hit flag as true. It also 
updates the access history of the page in the cache eviction policy, for example,
for LRU, it moves the page to the top of the list (of that tenant).

If the page is not in the cache, the algorithm checks if there is available space
in the cache. If there is, it gets the first available buffer location and adds the
page to the cache. 

If there is no available space, the algorithm selects a page to evict from the cache 
using the tenant selection policy and the cache eviction policy. The algorithm then 
evicts the page and adds the new page to the cache.

Note that the tenant selection policy's responsibility is to select the tenant to evict
from the cache, each time a page needs to be evicted, for this, some metrics of the 
tenants can be used.

The cache eviction policy's responsibility is to manage the cache, including:
\begin{itemize}
    \item \textbf{UpdateAccessHistory}(\textit{page}, \textit{tenant}): Update the access history when a page is accessed.
    \item \textbf{AddPageToCache}(\textit{page}, \textit{tenant}, \textit{buffer\_location}): Add a page to the cache of a tenant in a specific buffer location.
    \item \textbf{EvictPage}(\textit{tenant\_to\_evict}): Evict a page from the cache of a tenant.
\end{itemize}

Each of the described tenant selection policies and cache eviction policies will implement
these functions, and the base algorithm will call them when needed.

Note that the base algorithm is a general dynamic multi-tenant cache algorithm that can 
be used with different tenant selection and cache eviction policies. The next sections 
will describe the tenant selection and cache eviction policies used with the base algorithm.

Another algorithm for dynamic multi-tenant buffer allocation was proposed in the 
literature in \cite{buffer-sharing-1}, a hybrid approach combining global and static 
buffer allocation was proposed in \cite{article-for-2level-forecasting}.

\begin{figure}[htbp]
    \centering
    \begin{minipage}{\linewidth}
    \begin{algorithm}[H]
        \caption{Base Algorithm to process each page access}
        \begin{algorithmic}
            \STATE \textbf{Input:} $(p, t)$ - The page id and tenant of the page to be accessed.
            \STATE \textbf{Output:} $(b, h)$ - The buffer location of the page accessed and whether it is a hit or not.
            \STATE
            \STATE $(\text{buffer\_location}, \text{is\_hit}) \leftarrow \text{GetPageLocationIfItIsInCache}(\text{page} = p, \text{tenant} = t)$
            \IF {$\text{is\_hit}$}
                \STATE $\text{CacheEvictionPolicy.UpdateAccessHistory}(\text{page} = p, \text{tenant} = t)$
                \RETURN $(b = \text{buffer\_location}, h = \text{true})$
            \ELSE
                \STATE \COMMENT {Page is not in cache, we need to find a place for it}
                \IF {There is available space in cache}
                    \STATE $\text{available\_location} \leftarrow \text{GetAvailableBufferLocation}()$ \COMMENT {Get the first available buffer location}
                    \STATE $\text{CacheEvictionPolicy.AddPageToCache(\text{page} = p, \text{tenant} = t, \text{available\_location})}$
                    \RETURN $(b = \text{available\_location}, h = \text{false})$
                \ELSE
                    \STATE \COMMENT {Cache is full, we need to evict a page}
                    \STATE $\text{tenant\_to\_evict} \leftarrow \text{TenantSelectionPolicy.GetTenantToEvict}()$
                    \STATE $\text{evicted\_page} \leftarrow \text{CacheEvictionPolicy.EvictPage}(\text{tenant\_to\_evict})$
                    \STATE $b \leftarrow \text{evicted\_page.buffer\_location}$
                    \STATE $\text{CacheEvictionPolicy.AddPageToCache}(\text{page} = p, \text{tenant} = t, \text{buffer\_location} = b)$
                    \RETURN $(b, h = \text{false})$
                \ENDIF
            \ENDIF
            \STATE
            \STATE \textbf{function} GetPageLocationIfItIsInCache(page, tenant):
            \STATE \hspace{\algorithmicindent} \textbf{if} page is in cache \textbf{then}
            \STATE \hspace{\algorithmicindent} \hspace{\algorithmicindent} \textbf{return} $(\text{page.buffer\_location}, \text{true})$
            \STATE \hspace{\algorithmicindent} \textbf{else}
            \STATE \hspace{\algorithmicindent} \hspace{\algorithmicindent} \textbf{return} $(\text{null}, \text{false})$
            \STATE \hspace{\algorithmicindent} \textbf{end if}
            \STATE \textbf{end function}
        \end{algorithmic}
    \end{algorithm}
    \caption{Base Algorithm}
    \label{fig:base_algorithm}
    \end{minipage}
\end{figure}
    

\section{Tenant Selection Policy}

\lipsum[1-1] \ref{fig:figA-1}

\begin{figure}[H]
    \centering
    \includegraphics[width=0.3  \textwidth]{dummy.png}
    \caption{Lorem ipsum dolor sit amet}
    \label{fig:figA-2}
\end{figure}

\lipsum[1-1] \ref{fig:figB-1}

\begin{figure}[H]
    \centering
    \includegraphics[width=0.3\textwidth]{dummy.png}
    \caption{Lorem ipsum dolor sit amet}
    \label{fig:figB-2}
\end{figure}

\subsection{Fault Ratio Policy}

\lipsum[1-1] \cite{reference-1}

\subsection{Hit Ratio Policy}

\lipsum[1-1] \cite{reference-2}

\subsection{Fault Ratio with Cache Used Policy}

\lipsum[1-1] \cite{reference-3}

\subsection{Naive Policy}

\lipsum[1-1] \cite{reference-3}

\section{Cache Eviction Policy}

\begin{table}[ht!]
    \centering
    \begin{tabular}{c c c}
        \hline
        Column 1 & Column 2 & Column 3 \\
        \hline
        Item 1   & Item 2   & Item 3   \\
    \end{tabular}
    \caption{Lorem ipsum dolor sit amet}
    \label{tab:tabA-2}
\end{table}

\lipsum[1-1] \ref{tab:tabA-1}

\begin{table}[ht!]
    \centering
    \begin{tabular}{c c c}
        \hline
        Column 1 & Column 2 & Column 3 \\
        \hline
        Item 1   & Item 2   & Item 3   \\
    \end{tabular}
    \caption{Lorem ipsum dolor sit amet}
    \label{tab:tabB-2}
\end{table}

\lipsum[1-1]

\subsection{LRU}

\subsection{LFU}

\subsection{2Q}

\subsection{MQ}

\subsection{LIRS}

\subsection{LRU-2}

\subsection{LRFU}

\subsection{Bélády Optimum (for comparison)}

\lipsum[1-2]