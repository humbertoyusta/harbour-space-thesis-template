\chapter{Expermients and Methodology}

\section{Introduction}

This section will describe the experiments and methodology used to evaluate the proposed solutions. 
The experiments will be conducted on two different datasets: BrightKite and CitiBike, which are 
real-world datasets, and we will also include a synthetic dataset for parameter tuning.

\section{Datasets}

\subsection{BrightKite Data}

\subsubsection{Overview}

BrightKite was a location-based social network where users shared their locations by checking in at 
different places.

The dataset contains check-ins from users, and each check-in contains the user's ID, the location's ID, 
the timestamp, and the location's geographical coordinates.

BrightKite data was used by Thodoris Lykouris and Sergei Vassilvitskii in their paper "Competitive 
Caching with Machine Learned Advice" \cite{datasets-reference} and the dataset is publicly available
at \cite{brightkite-data}.

From this dataset, we created multiple instances of the caching problem, by considering each user 
as a tenant for our multi-tenant caching problem, location IDs as items or pages to be cached, and 
check-ins as item or page accesses. 

\subsubsection{Processing}

Similar to the work done by Thodoris Lykouris and Sergei Vassilvitskii in \cite{datasets-reference},
we selected the top users where the optimal cache algorithm (Belady's or clairvoyant) had more cache 
faults, to select the most challenging instances from the dataset.

Then we created 6 instances of the multi-tenant caching problem, by considering each of the most 
challenging users as a tenant, the locations as items to be cached, and the check-ins as item accesses.

We took 36 most challenging users from the dataset, and created 6 instances of the multi-tenant caching
problem, two of them with 4 tenants, two with 5 tenants, one with 8 tenants, and one with 10 tenants.

Information of the selected users, with number of accesses, number of unique items, and number of cache
faults for the optimal cache algorithm (with cache size equal to 50), rank and user ID, is shown in
Appendix \ref{appendix:brightike-top-36-users}.

The number of accesses per user is from 1300 to 2100, with most of them having 2100. The number of
unique items per user is from 300 to 1300, with many of them having around 600. The number of cache
faults for the optimal cache algorithm is from 10 to 164, with most of them having around 20 or 30.

To make the cache size recommendations per tenant, we started by some initial value, and kept increasing 
it until the number of cache faults for the optimal cache algorithm was less than some threshold, in this
case 40.

\subsection{CitiBike Data}

\subsubsection{Overview}

\subsubsection{Processing}

\subsection{Random Data}

\subsubsection{Overview}

\subsubsection{Generation}

\section{Testing and Validation}

\subsection{Data Validation}

\subsection{Algorithm Validation}

\section{Experiments}

\subsection{Hit Ratio Per Cache Size Experiment}

\subsection{Fault Score Experiment}

\lipsum[1-3]

\lipsum[1-1] \ref{fig:figA-1}

\begin{figure}[H]
    \centering
    \includegraphics[width=0.3  \textwidth]{dummy.png}
    \caption{Lorem ipsum dolor sit amet}
    \label{fig:figA-3}
\end{figure}

\lipsum[1-1] \ref{fig:figB-1}

\begin{figure}[H]
    \centering
    \includegraphics[width=0.3\textwidth]{dummy.png}
    \caption{Lorem ipsum dolor sit amet}
    \label{fig:figB-3}
\end{figure}


\lipsum[1-1] \cite{reference-1}


\lipsum[1-1] \cite{reference-2}


\lipsum[1-1] \cite{reference-3}


\begin{table}[ht!]
    \centering
    \begin{tabular}{c c c}
        \hline
        Column 1 & Column 2 & Column 3 \\
        \hline
        Item 1   & Item 2   & Item 3   \\
    \end{tabular}
    \caption{Lorem ipsum dolor sit amet}
    \label{tab:tabA-3}
\end{table}

\lipsum[1-1] \ref{tab:tabA-1}

\begin{table}[ht!]
    \centering
    \begin{tabular}{c c c}
        \hline
        Column 1 & Column 2 & Column 3 \\
        \hline
        Item 1   & Item 2   & Item 3   \\
    \end{tabular}
    \caption{Lorem ipsum dolor sit amet}
    \label{tab:tabB-3}
\end{table}

\lipsum[1-1] \ref{tab:tabB-1}

\lipsum[1-2]