\chapter{Conclusions and Recommendations}

\section{Conclusions}

Assuming reasonable seasonality recommendations for cache size per tenant, the proposed 
dynamic approach can achieve good results in enhancing cache performance and ensuring fair 
distribution among tenants in a multi-tenant environment.

The Fault Ratio with Cache Used Policy as a tenant selection policy, combined with different 
cache eviction policies, achieves comparable results to the baseline LRU solution with a 20\% 
higher cache size.

Several eviction policies were adapted to the multi-tenant dynamic cache environment and 
combined with the tenant selection policy. The adapted eviction policies were LRU, LFU, LRU-2, 
LIRS, 2Q, MQ, and LRFU. The LRU-2, LIRS, and MQ eviction policies consistently demonstrated 
strong performance across various datasets, outperforming classical LRU and LFU policies.

Experiments with real-world datasets, such as Brightkite and CitiBike, validated the 
robustness and versatility of the proposed solutions.

Proper hyperparameter tuning is crucial for achieving optimal performance. However, several 
solutions demonstrated strong performance across various datasets, both synthetic and 
real-world, using the same hyperparameters. The parameter tuning process was not overly 
complex, achieving good results with a very limited number of iterations.

\section{Recommendations}

The main focus on this thesis was on the performance of the proposed solutions after the 
seasonality recommendation was assumed. Future work should focus on evaluating the solution 
together with the seasonality recommendation algorithm. This will provide a more comprehensive
evaluation of the proposed solutions and their performance in real-world scenarios.

While the proposed solutions were evaluated using real-world datasets, these datasets were not 
multi-tenant caching problems in nature. Although these evaluations demonstrated 
the robustness and versatility of the proposed solutions, future work should focus on testing 
them within the specific context of multi-tenant caching problems. This can be achieved by 
using datasets from multi-tenant databases or applications to ensure more accurate and 
relevant assessments.

Synthetic data was added to facilitate generalized hyperparameter tuning due to the limited 
number of different real-world datasets. With a greater variety of real-world datasets, the 
need for synthetic datasets could be minimized or avoided altogether.

The main tenant selection policy used in this thesis was the Fault Ratio with Cache Used Policy. 
While other tenant selection policies were considered and tested, different approaches could be 
explored in future research.

Several eviction policies were adapted to the multi-tenant dynamic cache environment, but some 
policies, like ARC, are not trivial to adapt. Exploring approaches to adapt these policies or 
finding similar policies that can be adapted is an important area for further investigation.

No machine learning solutions were evaluated as cache eviction policies in this thesis; 
however, they could be an interesting approach to explore in future research.

2Q eviction policy did not perform well in the experiments, while \cite{2q-article} claims to 
achieve similar results to LRU-2, which was one of the best-performing eviction policies in
this thesis. This discrepancy could be due to the adaptation of the eviction policy to the
multi-tenant environment, specific implementation details, parameter tuning, or the datasets
used in the experiments. Further investigation is needed to understand the reasons behind this
discrepancy.

